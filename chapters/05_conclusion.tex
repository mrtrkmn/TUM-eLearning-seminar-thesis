\chapter{Conclusion}

Haaukins platform initiated with support from non-profit organizations mainly Industriens Fond\footnote{https://industriensfond.dk} in Denmark and developed by Aalborg University. Although existing platforms might give some flexibility at some extend, they do not provide full freedom in terms of usage. The existing platforms are also decent, has enormous amount of resources however they have different goals compared to non-profit platform, Haaukins. 


The main aim is to help students who are struggling with understanding basics of lectures primarily by targeting high-school students. The platform showed itself with its success at high schools changed its target group to include university students. As the platform is evolved in time, the requirements of the platform changed according to user feedback comes from students and instructions. Participants of an event on the platform wanted to have more manageable environment by providing them an access to manage their environment. The management from students side is given and limited to their own virtualized environment. Even though they do not have any influence on any running other virtual environments, they can restart their virtual environment when there is a problem. The administration side of the platform is re-designed and built from scratch with user-friendly, easy to manage and informative features in mind, explained in the management chapter.  All the improvements have been done on the platform to deliver pleasure time while learning both for students.
The growing interest on security education is pushing the platform evolve more in the future, including more advanced features. The platform\footnote{https://github.com/aau-network-security/haaukins} is in active development at Github and generally available on event "General"\footnote{https://general.haaukins.com}. Since it is open to anyone who is interested to do practical experience, anyone can sign up at no cost. 
Further requests and improvements are very welcome by the community, it is possible to initiate a discussion from main Github repository under discussions.\footnote{https://github.com/aau-network-security/haaukins/discussions}
    